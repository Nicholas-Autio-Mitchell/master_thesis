% Mathekram

\chapter{Mathekram}
\label{mk}

%%%%%%%%%%%%%%%%%%%%%%%%%%%%%%%%%%%%%%%%%%%%%%%%%%%%%%%%%%%%%%%%%%%%%%%%%%%%%%%
%%%%%%%%%%%%%%%%%%%%%%%%%%%%%%%%%%%%%%%%%%%%%%%%%%%%%%%%%%%%%%%%%%%%%%%%%%%%%%%

\section{Gleichungen}

Hin und wieder wird man eine Gleichung schreiben müssen. Wie diese z.B.
\begin{equation}
	f(x)
	= x_i^{234} \hspace{1mm} \frac{a}{c+d}
		\int_0^l\limits{
		g(y) \ \frac{\partial h(z(y))}{\partial z}
		}{\rm d} y
	\label{mk_bla}
\end{equation}
Auf Gleichungen, die einen Label bekommen haben, kann man referenzieren, z.B. ich finde die Gleichung \eqref{mk_bla} super!

Die Gleichungen in der \verb=\equation=-Umgebung werden automatisch nummeriert, genauso so wie alle Kapitel, Sektionen und Untersektionen (ohne Stern). Entsprechend kann man nicht nummerierte Gleichungen mit 
\begin{equation*}
	f(x)
	= x_{ijkllmnopqrst}^{234} \hspace{1mm} \frac{a}{c+d}
		\int_0^l\limits{
		g(y) \ \frac{\partial h(z(y))}{\partial z}
		}{\rm d} y
\end{equation*}
erstellen.

%%%%%%%%%%%%%%%%%%%%%%%%%%%%%%%%%%%%%%%%%%%%%%%%%%%%%%%%%%%%%%%%%%%%%%%%%%%%%%%
%%%%%%%%%%%%%%%%%%%%%%%%%%%%%%%%%%%%%%%%%%%%%%%%%%%%%%%%%%%%%%%%%%%%%%%%%%%%%%%

\section{Mathe im Text}

Manchmal wird man auch mitten im Text ein Stückchen Mathe haben wollen oder eine definierte Größe wie zum Beispiel $f(x)$ schreiben wollen. Hierzu verwendet man die Mathemodus \verb=$$= im Text, wie auch hier $g(x) = x^7 + a_i$.

%%%%%%%%%%%%%%%%%%%%%%%%%%%%%%%%%%%%%%%%%%%%%%%%%%%%%%%%%%%%%%%%%%%%%%%%%%%%%%%
%%%%%%%%%%%%%%%%%%%%%%%%%%%%%%%%%%%%%%%%%%%%%%%%%%%%%%%%%%%%%%%%%%%%%%%%%%%%%%%

\section{Matrizen}

Hier gibt es viele Möglichkeiten, eine Matrix bzw. ein Array darzustellen, wie z.B.
\begin{equation}
	\begin{array}{lrc}
	12 & 1234 & 2 \\
	\mbox{Dies ist einfach Text} & Text im Array & 234567 \\
	\frac{345}{4}
	\end{array}
\end{equation}
\begin{equation}
	\left(
	\begin{array}{lrc}
	12 & 1234 & 2 \\
	\mbox{Dies ist einfach Text} & Text im Array & 234567 \\
	\displaystyle \frac{345}{4}
	\end{array}
	\right)
\end{equation}
\begin{eqnarray}
	\begin{pmatrix}
	12 & 1234 & 2 \\
	\mbox{Dies ist einfach Text} & Text im Array & 234567 \\	
	\end{pmatrix}
	\\ % Nächste Gleichung
	\begin{bmatrix}
	12 & 1234 & 2 \\
	\mbox{Dies ist einfach Text} & Text im Array & 234567 \\	
	\end{bmatrix}
\end{eqnarray}

% Pakete

\usepackage[utf8]{inputenc} % zur Erkennung von ä,ö,ü und ß (funktioniert nur, wenn ALLE *.tex-Dateien in UTF8 kodiert sind)
\usepackage{graphicx} % benötigt für \includegraphics und EPS-Figuren
\usepackage{amsmath} % benötigt für \eqref 
\usepackage[colorlinks=true,ngerman,breaklinks=true]{hyperref} % benötigt für \autoref und klickbare Hyperlinks, ngerman für deutsche Referenzen, colorlinks färbt die Referenzen (Boxen weg), breaklinks bricht lange Referenzen wie in Abbildungsverzeichnis
\usepackage{xcolor}
\definecolor{c1}{rgb}{0.8,0,0.4}
\definecolor{c2}{rgb}{0,0.3,0.4}
\definecolor{c3}{rgb}{0,0,0.7}
\hypersetup{
	linkcolor=c3, % interne Links
	citecolor=c2, % Zitate, Literatur
	urlcolor=c1 % externe Links
}
\usepackage{enumerate} % für die Freigaben der Einstellung von enumerate und itemize
\usepackage{longtable} % erlaubt die Benutzung von longtable-Umgebung
\usepackage{bigstrut} % benötigt für \bigstrut 
\usepackage{psfrag} % benötigt für \psfrag 
\usepackage{ngerman} % benötigt für alle automatisch generierte Worte bis auf Referenzen
\usepackage[round]{natbib} % benötigt für \cite 
\usepackage{makeidx} % benötigt für die Erstellung vom Index
\makeindex
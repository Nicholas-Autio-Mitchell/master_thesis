


\documentclass{pracjourn}
\usepackage{fancyvrb}

%% Revision Control
\TPJrevision{2007}{28}{12} %% First Draft


%% Mandatory Data
\title{Using \LaTeX\ for writing a Thesis}
\author{Rohit Vishal Kumar}
\abstract{\LaTeX\ has been successfully used for typesetting widely different
document format.  However, the complexity of typesetting some commonly used
documents, normally acts as a deterrent for various people, who would like to
use \LaTeX\ for their work. Over the years, I have noticed that students who
come to \LaTeX, eager in their anticipation of using \LaTeX, loose their
enthusiasm, midway, and revert back to using MS Word. In this article, I have
tried to described my own experiences of typesetting a doctoral thesis using
widely available packages; in the hope that students can see the the ease with
which \LaTeX\ can be used for complex work.}


%% Additional Data
\email{rohitvishalkumar@yahoo.com}
\address{Department of Marketing \\
Xavier Institute of Social Science \\
Ranchi, 834001, Jharkhand, India}
\license{Copyright \textcopyright\ 2007 Rohit Vishal Kumar}


%% Begin Writing the Document
\begin{document}
\maketitle


\section{Introduction}

A Thesis or a dissertation is probably the most complex document that a student
of higher  studies is compelled to write at some point of his academic career.
The complexity of thesis arises from various factors some of which are beyond
the control of student and have to be taken for granted. For example, the
discipline in which the student is pursuing his studies may require use of
special characters, shapes or formulas which may be difficult to typeset in a
word processor. The university in which the student is enrolled may have its
own requirements as to how to format the various parts of the thesis. Besides
these, the viewpoint put forward by the student in the thesis may have its own
peculiar requirements. Furthermore, the thesis normally requires that
bibliography be provided in a particular style. Social sciences, for example,
mostly require in-line citations where as more rigorous science may well
require numerical citation.

So when a student first sits down to write the thesis, his first choice is the
word processor with which he  is most familiar with. Unfortunately, word
processors are normally not designed for writing thesis. Their forte lies in
writing official documents like letters, small reports etc. \dash documents
which do not require complex formatting; but which may require a plethora of
eye candy.

\LaTeX\ is a \TeX\ macro package, originally written by Leslie Lamport, that
provides a document  processing system \cite{LAMPORT}. It allows markup to
describe the structure of the document and the underlying \TeX\ program reads
the markup and format the document accordingly. This allows for significant
advantage, in the sense, that the person can focus more or content, than on
formatting the document. Beside this generic advantage, various other
advantages are offered by \LaTeX. The \BibTeX\ program helps to insert the
citation's in the document and provides the bibliography in almost any desired
format. \LaTeX\ can also be used for creating glossaries and indexes and of
course, \LaTeX\ excels in typesetting mathematical formulas. There are other
advantages also \dash the \LaTeX\ source code is a text file which is editable
on almost any computer and operating system and does not run the risk of being
corrupted with viruses. This makes the task of making backup simpler. Larger
documents in \LaTeX\ can be split up into smaller parts, which can then be
inserted into the correct place with the minimum of effort. This also allows
the user the flexibility to work on individual chapters, rather than the whole
document at one go. Finally, \LaTeX\ is adept at producing \PS\ (PS) or
Portable document File (PDF) \dash which can be printed on any printer without
any loss of formatting.

Perhaps the paean of \LaTeX\ may jar on some ear's, specially of the students
used to word processors, but the jackpot  at the end of the day is really huge.
This article assumes that the reader has rudimentary knowledge of \LaTeX\ and
understands the basics. In the next section,  we take a look at the parts of
the thesis and subsequently we take a look at how to format a basic thesis
which should meet the requirement of most universities.

In what follows, we assume the reader is already familiar with the basics of
\LaTeX\ and we shall use freely the \LaTeX\ jargon.

\section{A review of current solutions}

Let us now take a look at how to write a thesis using \LaTeX. A simple search
of the Comprehensive  \TeX{} archive Network (CTAN) produces a fairly large
number of thesis packages which can be used by the beginner in \LaTeX. For
example, \texttt{muthesis} \cite{MUTHESIS} aimed at University of Manchester,
Department of Computer Science, thesis style;   \texttt{ut-thesis}
\cite{UTTHESIS1} aimed at University of Toronto thesis style; \texttt{utthesis}
\cite{UTTHESIS}for University of Texas (Austin), \texttt{uwthesis}
\cite{UWTHESIS} for University of Washington thesis stye; \texttt{uaclasses}
\cite{UACLASSES} for University of Arizona theis style. The above classes can
be found at \ctanloc{/Formats/LaTeX/Contrib/}. Searching for other thesis
classes, some more were found at the \ctanloc{/Uncategorized} section, namely
\texttt{thutthesis} (Tsinghua University), \texttt{stellenbosch} (University of
Stellenbosch), \texttt{nddiss} (University of Notre Dame), \texttt{fbithesis}
(University of Dortmund) \texttt{toptesi} (Polytechnique of Turin) etc. As can
be seen, almost all of the thesis classes are geared to meet the requirements
of a particular university. Any of these can be used as a starting point and
then customised for writing the thesis. However, that may not be feasible for a
new person who has just started using \LaTeX.

Some other packages which are more flexible and generic in nature are  reviewed
here:  (i) \texttt{hepthesis} \cite{ABUCKLEY} a special thesis class aimed at
writing thesis in the field of high energy particle physics. The package is
flexible enough to be used as a general thesis package, but is dependent on
various other packages which are available on CTAN. The package provides
significant advantages without being too complex in nature. The problem is that
if some non-standard specifications are required then the student may not have
an easy way of making the changes. For example, the package provides for
predefined margins and any changes are not easily made. (ii)
\texttt{classicthesis} \cite{AMIEDE} is another package which provides the
first time users with an easy way of writing the thesis. However, the
typographical formatting is done in homage to Robert Bringhurst book ``The
Elements of Typographic Style'' and is overtly classical in nature. Another
point of dispute is that it presumes that the thesis is a book and uses a wide
outer margin on the right and the left hand pages. Customisation seems to be a
problem, as the author himself requests the user not to change the style
\cite[pp. 2]{AMIEDE}. (iii) \texttt{jkthesis} \cite{JKUPPER} seems to be
another generic thesis class. However as the documentation was in German, I
could not do justice to the class. Prima Facie it seems to be reasonably
flexible class for preparation of thesis.

Besides the above, some other classes were also found, but the documentation
was not  enough to form a definite opinion. Almost all the solutions given
above seems to be hardwired to a particular university or a style and were not
flexible enough, from the point of view of a layman. In search of something
more flexible, I made two  assumptions that the thesis is not a book but a
report and that there should be flexibility in adopting the thesis to ones
needs.

\section{Using \LaTeX{} for Thesis Writing}

The thesis can be looked upon as consisting of three basic parts which are
discussed below.

The first part is the \textbf{Front Matter} In the front matter comes various
parts which can be listed as follows : (i) Title page - which is in most of the
cases also the cover page of the thesis (ii) Table of Contents (iii) List of
tables and (iv) List of Figures. A dedication page or a quotation page can also
be included.

The second and the longer part is the \textbf{main matter} The main matter
would consist of the chapters of the thesis such as (i) Introduction (ii)
Problem statement (iii) review of literature (iv) methodology (v) data analysis
(vi) findings and (vii) conclusion. From the viewpoint of using \LaTeX{},
appendicies can also be included in the main matter. This is where the student
would develop the rationale for undergoing the research work and defend his or
her findings. It is in this section, the student would include the mathematical
equations, charts and graphs and other illustrations.

Finally at the end comes the \textbf{Back Matter}. This is the tail of the
thesis and  consist of some things which are extremely difficult to produce
using a normal word processor. This normally consist of (i) Bibliography or
References (ii) Glossary and (iii) Index. \LaTeX{}, which was designed for
typesetting takes care of the various aspects of the back matter in a more
efficient way than any word processor and saves a lot of time and energy for
the researcher.

\subsection{Packages and Assumptions}

The first assumption that I made is that the thesis is a report and not a book.
This is because, almost universally, the thesis is printed on only the right
hand side of a page. Secondly, thesis requires that one-half or double spacing
be followed while typing the matter. And thirdly, it should conform to the
margin requirements of the page setup. Most universities require that the left
margin include a gutter or space for binding.

Based on these assumptions, I have selected the following packages which are
needed for our task  \dash \texttt{url}, \texttt{setspace} and
\texttt{geometry}. The \texttt{url} package is a much needed package which
helps in formatting the long web url's, email id's etc. properly both in the
document and in the bibliography. The usage is extremely simple and any url
which needs to be typeset just needs to be enclosed in a \verb|\url{...}|
command.  The \texttt{setspace} package provides a much easier method of
controlling spacing in the document. Where ever we need a single spacing the
command \verb|\singlespacing| does the job. Similarly \verb|\onehalfspacing|
and \verb|doublespacing| provides easy switch based mechanism to turn on or off
the style of spacing desired. I made acquaintance with the \texttt{geometry}
package much later in my thesis work but I wished I had done it sooner. It is
an extremely powerful package which provides numerous options to control the
page layout.

\section{Laying out the Front Matter}

\subsection{Preamble}

Given the above assumptions, and the objective to provide the user with control over the flexibility, the preamble looked as follows:

\begin{Verbatim}[frame=lines, xleftmargin=2mm, framesep=2mm, numbers=left]
\documentclass[12pt, a4paper]{report}
\usepackage{graphicx}
\usepackage{url}
\usepackage{setspace}
\usepackage[left=1.2in, right=1in, top=1in, bottom=1in]{geometry}
\begin{document}
\pagenumbering{roman}
\end{Verbatim}

The standard 12 point type is used on a4paper. Two additional package seen
\dash namely the \texttt{graphicx} ---  is the well know package for inclusion
of graphics in to the final document.  The use of \verb|\pagenumbering{roman}|
changes the default page numbering to small roman numerals in the front matter
of the thesis.

\subsection{Title Page}

The title page normally requires five elements --- the title of the thesis, the
degree for which  the thesis has been submitted, the name of the author, the
department in which the work was undertaken and the year of submission. To
achieve this, we have provided a flexible alternative which is easy to
understand and can be customized by anyone to meet his own requirement.

\begin{Verbatim}[frame=lines, xleftmargin=2mm, framesep=2mm, numbers=left, firstnumber=last]
\begin{titlepage}
\vspace*{\stretch{1}}
\begin{center}
\Huge{\textsc{Title of the Thesis}} \\
\vspace{4em}
\large{Thesis Submitted To} \\
\vspace{2em}
\Large{\textsc{Name of the University}} \\
\vspace{1em}
\large{in partial fulfilment of the requirements} \\
\large{of the award of the degree} \\
\vspace{2em}
\Large{\textsc{Name of the Degree}} \\
\vspace{2em}
by \\
\vspace{1em}
\textbf{\textsc{Name of Candidate}} \\
\vspace{4em}
\Large{\textsc{Name of the Department}} \\
\Large{\textsc{Name of the University}} \\
\Large{Year}
\end{center}
\vspace*{\stretch{1}}
\end{titlepage}
\end{Verbatim}

This approach to title page design provides flexibility into the hands on
newbie. The new user can also  customize the cover page to meet the requirement
of his university by inserting logo's, or incorporating relevant information.
The generous use of spaces (\verb|\vspace{}|) allows the user to customize the
placements of the various elements on the title page.

\subsection{Other Elements of Front Matter}

The other elements that would require to appear in the front matter are the
table of contents, tables and figures.  This process is fairly straightforward
and uses the standard \LaTeX\ commands.

\begin{Verbatim}[frame=lines, xleftmargin=2mm, framesep=2mm, numbers=left, firstnumber=last]
\tableofcontents
\listoftables
\listoffigures
\newpage
\end{Verbatim}

This brings us to the end of setting up the front matter of the thesis.

\section{Laying out the Main Matter}

Next, we turn our attention to designing the main matter. The main body of the
thesis consists of the chapters and annexes, which are generally typeset in
one-half or double spacing. The page numbering used is the Arabic numeral
system and this continues till the end of the thesis.

Laying out the main matter, is perhaps, the easiest task in \LaTeX\ and is
accomplished by dividing the various  chapters into individual files. I have
assumed that each chapter is broken into a file. Each of these chapter files
being with the command \verb|\chapter{....}| and then has the necessary
material which needs to be typeset; except for the ``acknowledgement'' which
starts with the starred version of the chapter command viz
\verb|\chapter*{....}|. This is because, we don't want the acknowledgement to
have a chapter number.

Under the assumption that there are five chapters and three annexes, we can set
up the main body as follows:

\begin{Verbatim}[frame=lines, xleftmargin=2mm, framesep=2mm, numbers=left, firstnumber=last]
\doublespacing
\include{CHAP_00}           % 0. Acknowledgement
\pagenumbering{arabic}
\part*{T H E S I S}
\include{CHAP_01}           % 1. Introduction
\include{CHAP_02}           % 2. Literature Review
\include{CHAP_03}           % 3. Research Methodology
\include{CHAP_04}           % 4. Findings and Outputs
\include{CHAP_05}           % 5. Conclusions
\part*{A P P E N D I X}
\appendix
\include{APD_01}            % Appendix I
\include{APD_02}            % Appendix II
\include{APD_03}            % Appendix III
\end{Verbatim}

This is all there is to set up the main body of the thesis. All a  newbie has
to take care is that he should remember to change the page numbering back into
arabic. The \verb|\part{}| command, in my opinion are purely decorative and may
be dropped without much loss of formatting. However, keeping them in place
produces a separator page with ``T H E S I S'' and ``A P P E N D I X'' written
in the center, which help in demarcating the appendix from the main set of
chapters.

\section{Laying out the Back Matter}

We are almost at the end of our thesis layout. In the back matter, I have made
a simplifying assumption  that the student has only to submit the bibliography.
Bibliography can either be set up within the document or it can be done as a
separate file. To produce the bibliography within the document we normally use
the \verb|\thebibliography| environment.

As this method is cumbersome and does not lead to easy use with the various
bibliographical packages, this method  is not recommended. It is always much
better to go in for a separate bibliography file and the best way to create a
bibliography file would be to use a graphical bibliography manager such as
Jabref (\url{http://jabref.sourceforge.net/}). Once the user has created the
bibliography file, it needs to be inserted in the document.

\begin{Verbatim}[frame=lines, xleftmargin=2mm, framesep=2mm, numbers=left, firstnumber=last]
\newpage
\footnotesize
\singlespacing
\bibliographystyle{plain}
\bibliography{mybib}
\end{document}
\end{Verbatim}

The \verb|\newpage| command forces \LaTeX{} to insert a new page. The
\verb|\footnotesize| changes the default  font size to a smaller font size and
the \verb|\singlespacing| turns on the single spacing for the bibliography. The
\verb|\bibliographystyle{plain}| tells \LaTeX{} to use the \texttt{plain}
bibliographical style.

\LaTeX{}, by default provides the following four styles \dash \texttt{alpha}
for producing a bibliography which  is sorted alphabetically, the labels being
formed from the name of the author and the year of publication, \texttt{unsrt}
for producing bibliography similar to the \texttt{plain} style, but the entries
are in the order of citation, \texttt{abbrev} for producing the abbreviated
bibliography and \texttt{plain} for producing the alphabetically sorted with
numerical label styles. The command \verb|\bibliography{mybib}| tells \LaTeX{}
to use the bibliographical file named \texttt{mybib.bib}. More than one
bibliography file can be used by passing on the names as arguments as shown:
\verb|\bibliography{file1,file2,...,fileN}|.

For getting the author-date citation style, more common in social sciences, it
requires additional packages.  Some packages that can be used to provide
``author-date'' citation style are \texttt{apacite} and \texttt{natbib}. All it
requires is that one additional command be used in the preamble like \dash
\verb|\usepackage{apacite}|. As each of the packages have different ways of
usage and modify the default \verb|\cite{}| command, it is suggested that the
student familiarize himself with the package before using it.


\section{In Sum}

This bring us to an end of trying to write a thesis in \LaTeX{}. The above
format is  generic enough to meet many requirements and is flexible enough to
meet the need's of the advanced user and specific subjects. Packages desired
for a specific discipline can be included in the preamble. Another major
advantage of such an approach is that the students can work on any chapter of
their choice and compile only the desired chapter. Say, for example, we are
working on the chapter 3 of our hypothetical thesis. We can only compile this
chapter by issuing the command \verb|\includeonly{CHAP_03}| in the preamble.

The included zip file contains the skeleton structure of the thesis, along with
the chapters. The bibliography of this article has been included. It is hoped
that this article will provide encouragement to the students writing their
thesis and encourage them to use \LaTeX{} for the same.

%%%%%%%%%%%%%%%%%%%%%%%%%%%%%%%%%%%%%%%
%\newpage
\bibliographystyle{alpha}
\bibliography{mybib}
\end{document}

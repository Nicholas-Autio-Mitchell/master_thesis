% Created 2016-01-08 Fri 17:08
\documentclass{article}
\usepackage[utf8]{inputenc}
\usepackage[T1]{fontenc}
\usepackage{fixltx2e}
\usepackage{graphicx}
\usepackage{longtable}
\usepackage{float}
\usepackage{wrapfig}
\usepackage{rotating}
\usepackage[normalem]{ulem}
\usepackage{amsmath}
\usepackage{textcomp}
\usepackage{marvosym}
\usepackage{wasysym}
\usepackage{amssymb}
\usepackage{hyperref}
\tolerance=1000
\author{Nicholas Mitchell}
\date{\today}
\title{Report Structure}
\hypersetup{
  pdfkeywords={},
  pdfsubject={},
  pdfcreator={Emacs 24.5.1 (Org mode 8.2.10)}}
\begin{document}

\maketitle
\tableofcontents

\pagebreak

This file: A rough outline to the thesis. Each heading below may be separated into its own .tex file (in folder "content")


\section{Introduction}
\label{sec-1}

\begin{enumerate}
\item Overview of research area
\label{sec-1-1}

\item Motivation for thesis
\label{sec-1-2}
An introduction: what are we setting out to better understand? Where can we shed light within the current state of prediction making in a financial context using social media data?
On top of incorporating social media data, which specific/special statistical methodology are we implementing, and why?

\item Literature review (combined with the motivations?)
\label{sec-1-3}
An overview of similar projects using social media data to effect (Google work from Okhrin's chair).
Work using our statistical methodology, a jsutification why it is fitting for our model (boosting allows the inclusion of many factors without necessarily watering down the model at the same time)

\item Thesis breakdown
\label{sec-1-4}

\begin{enumerate}
\item The order of the thesis -> try to tell a story that can be read from start to finish, each step easily comprehendable.
\label{sec-1-4-1}

\item List hypotheses?
\label{sec-1-4-2}
\end{enumerate}
\end{enumerate}


\section{Data}
\label{sec-2}

\begin{enumerate}
\item Data Overview
\label{sec-2-1}

\begin{enumerate}
\item What data do we aim to use? Here a brief overview, but a section later on explaining how the data was collected and prepared for modelling.
\label{sec-2-1-1}

\item What is our justification for this data?
\label{sec-2-1-2}
\end{enumerate}


\item Twitter Mining
\label{sec-2-2}

\begin{enumerate}
\item Overview [Background for reader but also for future reference for DEVnet]
\label{sec-2-2-1}

\begin{enumerate}
\item The end goal
\label{sec-2-2-1-1}
We want data that looks like \textbf{this} which can be edited like \textbf{this} and is reliable, reproducible, relevant, \ldots{}
\item The challenges
\label{sec-2-2-1-2}
There are several sources of Twitter data, each with their own strengths and shortcomings. Below is a summary of each.
\end{enumerate}

\item Twitter API for Developers
\label{sec-2-2-2}

\begin{enumerate}
\item What is it?
\label{sec-2-2-2-1}

\item How does it work?
\label{sec-2-2-2-2}

\item Advantages
\label{sec-2-2-2-3}

\item Limitations
\label{sec-2-2-2-4}
\end{enumerate}

\item Third party companies
\label{sec-2-2-3}

\begin{enumerate}
\item What is it?
\label{sec-2-2-3-1}

\item How does it work?
\label{sec-2-2-3-2}

\item Advantages
\label{sec-2-2-3-3}

\item Limitations
\label{sec-2-2-3-4}
\end{enumerate}

\item Twitter advanced search
\label{sec-2-2-4}

\begin{enumerate}
\item What is it?
\label{sec-2-2-4-1}

\item How does it work?
\label{sec-2-2-4-2}

\item Advantages
\label{sec-2-2-4-3}

\item Limitations
\label{sec-2-2-4-4}
\end{enumerate}

\item How we have used the advanced search
\label{sec-2-2-5}

\begin{enumerate}
\item Advantages vs. Disadvantages
\label{sec-2-2-5-1}
\end{enumerate}
\end{enumerate}


\item\relax [Optional] Scraping with Python
\label{sec-2-3}

\begin{enumerate}
\item Overview
\label{sec-2-3-1}
Explain general methodology, difficulties and their solutions
This may be better as a larger appendix
\end{enumerate}

\item\relax [Include in appendix?] Data Preprocessing
\label{sec-2-4}

\begin{enumerate}
\item Features of the Twitter data
\label{sec-2-4-1}
Possibly talk about any differences between our scraped data and the data that is available from the API
\item Our final version of Twitter data
\label{sec-2-4-2}
A simple example table of the final version that gets imported into R
\end{enumerate}

\item Inspection of Entire Data Set
\label{sec-2-5}

\begin{enumerate}
\item Starting point for modelling
\label{sec-2-5-1}

Having collected and cleaned the Twitter data, the next step is to look at it in context, alongside the common and more directly related financial market data
It is important that some level of correlation is present.
Here we could explain our hypotheses (mentioned in first section):
\begin{itemize}
\item Frequency of tweets is telling of near future market movements?
\item Or rather it may give us a measure of momentum? A lot of tweets don't tell us how the direction will change, but rather hopw long it may stay on its present course.
\item Can we measure a general delay between market movements and the respose on Twitter? (Maybe it does indeed run in the opposite direction?)
\end{itemize}
\end{enumerate}
\end{enumerate}


\section{Sentiment Analysis}
\label{sec-3}

\begin{enumerate}
\item Introduction
\label{sec-3-1}

\begin{enumerate}
\item What is sentiment analysis and why can it help us to model the markets
\label{sec-3-1-1}
\end{enumerate}

\item Models to be applied
\label{sec-3-2}

\begin{enumerate}
\item SentiStrength, Emolex, Sentinet140, Vader Afinn, Vader
\label{sec-3-2-1}

\item Short explanation of each of the five models used:
\label{sec-3-2-2}

\begin{enumerate}
\item the underlying philosophy
\label{sec-3-2-2-1}

\item the algorithm
\label{sec-3-2-2-2}

\item understanding the output
\label{sec-3-2-2-3}
\end{enumerate}
\end{enumerate}
\end{enumerate}


\section{Boosting}
\label{sec-4}

\begin{enumerate}
\item Theoretical background
\label{sec-4-1}

\begin{enumerate}
\item Friedman - sequential regression, using residuals to fit next learner
\label{sec-4-1-1}

\item Parameters: number of iterations, shrinkage (learning rate), tree depth. For each, explain:
\label{sec-4-1-2}

\begin{enumerate}
\item importance - how can it affects/helps refine results
\label{sec-4-1-2-1}

\item limitations - what happens if we get parameters wrong or, for example, had infinite time to compute things?
\label{sec-4-1-2-2}
Where are the bottle necks? Why are other models better in certain situations?
\end{enumerate}
\end{enumerate}

\item Strengths \& Weaknesses
\label{sec-4-2}

\begin{enumerate}
\item Number of covariates is no longer an issue (assuming we hae enough iterations)
\label{sec-4-2-1}

\item (Multi)collinearity isn't such a worry, as the most important predictors are used - in addition we can 'prune' early on
\label{sec-4-2-2}

\item Collinearity isn't a problem perhaps anyway in a strict sense as there is a lot of noise in our data sets - see the comments on: \url{http://stats.stackexchange.com/questions/30903/what-does-that-mean-that-two-time-series-are-colinear}
\label{sec-4-2-3}

\item Can be optimised according to any given loss function (Least squares, absolute error, Huber error, \ldots{})
\label{sec-4-2-4}
These can be tailor-fitted to data. If we believe the data set to be non-Gaussian, a different loss-function can be used.

\item The sequential learning steps can be performed stochastically to increase model performance AND computation times
\label{sec-4-2-5}

\item 
\label{sec-4-2-6}
\end{enumerate}

\item Why does it suit the requirements of this research?
\label{sec-4-3}

\begin{enumerate}
\item We have many predictors, meaning the dimensions of the data [e.g. 670 x 400] are not typically great time-series/predictive analysis
\label{sec-4-3-1}

\item \ldots{} Compare to other models that are not suited to this data?
\label{sec-4-3-2}
\end{enumerate}
\end{enumerate}


\section{Description of our model}
\label{sec-5}

\begin{enumerate}
\item Define our boosting model:
\label{sec-5-1}

\begin{enumerate}
\item How are the model parameters optimised
\label{sec-5-1-1}

\item Cross validation
\label{sec-5-1-2}

\item Loss function plots
\label{sec-5-1-3}

\item AUC/ROC plots?
\label{sec-5-1-4}
\end{enumerate}

\item Principal Component Analysis / (one other common and robust techniques to analyse data?)
\label{sec-5-2}

\begin{enumerate}
\item Can we show that the addition of sentiment analysis data improves the ability of the model to explain variance?
\label{sec-5-2-1}
Is it possible that Sentiment analysis would add another dimension to the PCA (or SCA)
\end{enumerate}
\end{enumerate}


\section{Data subsets}
\label{sec-6}

\begin{center}
\begin{tabular}{lll}
Name & Components (log$_{\text{ret}}$ \textasciitilde{} \ldots{}) & Reasoning\\
\hline
dow$_{\text{only}}$ & lagged log$_{\text{ret}}$ & Most basic example for comparison\\
dow$_{\text{trad}}$ & gold, oil, sp500, int$_{\text{rates}}$ & traditional model factors\\
dow$_{\text{macro}}$ & all macro data & Many macro factors handled well\\
dow$_{\text{SA}}$$_{\text{avg}}$ & average sentiment scores & Sentiment analysis explains var.\\
dow$_{\text{SA}}$$_{\text{all}}$ & all individual SA results & SA from certain models might perform better\\
 &  & \\
\hline
dow$_{\text{best}}$ & trad + macro + best of SA & All data to showcase component-wise boosting\\
\hline
\end{tabular}
\end{center}


\section{For each model:}
\label{sec-7}

\begin{enumerate}
\item Explanation/Reasoning behind the model
\label{sec-7-1}

\item Analysis of Results
\label{sec-7-2}
\end{enumerate}


\section{Discussion of all results / comparison to literature}
\label{sec-8}

\section{Further work}
\label{sec-9}

\begin{enumerate}
\item Other sources of social media data
\label{sec-9-1}

\item Extensions to the model:
\label{sec-9-2}

\begin{enumerate}
\item Use of PCA to capture the variance of model with fewer predictors
\label{sec-9-2-1}

\item This may allow boosting to run for longer and so in sum produce better results
\label{sec-9-2-2}
\end{enumerate}
\end{enumerate}
% Emacs 24.5.1 (Org mode 8.2.10)
\end{document}
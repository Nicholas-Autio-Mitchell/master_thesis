% Created 2016-03-08 Tue 00:16
\documentclass{article}
\usepackage[utf8]{inputenc}
\usepackage[T1]{fontenc}
\usepackage{fixltx2e}
\usepackage{graphicx}
\usepackage{longtable}
\usepackage{float}
\usepackage{wrapfig}
\usepackage{rotating}
\usepackage[normalem]{ulem}
\usepackage{amsmath}
\usepackage{textcomp}
\usepackage{marvosym}
\usepackage{wasysym}
\usepackage{amssymb}
\usepackage{hyperref}
\tolerance=1000
\usepackage{minted}
\author{Nicholas Mitchell}
\date{\today}
\title{Test report}
\hypersetup{
  pdfkeywords={},
  pdfsubject={},
  pdfcreator={Emacs 24.5.1 (Org mode 8.2.10)}}
\begin{document}

\maketitle
\tableofcontents


\section{Example of Org-Babel for R Literate Programming}
\label{sec-1}
\begin{enumerate}
\item R text output
\label{sec-1-1}
A simple summary. 
\begin{minted}[]{r}
x <- rnorm(10)
summary(x)
\end{minted}

\item R graphics output
\label{sec-1-2}
Note we use the object \texttt{x} generated in previous code block, thanks to
the header option \texttt{:session *R*}.  The output graphics file is
\texttt{a.png}. 

\begin{minted}[]{r}
y <- rnorm(10)
plot(x, y)
\end{minted}

Same plot with larger dimension:

\begin{minted}[]{r}
plot(x, y)
\end{minted}

\begin{minted}[]{r}
for(i in 1:seq(1, 3)) {

    print(i)
}
\end{minted}

\begin{minted}[linenos,firstnumber=1]{python}
_def fun(input):
     input + 1
\end{minted}


\begin{verbatim}
1  something here
\end{verbatim}
\end{enumerate}
% Emacs 24.5.1 (Org mode 8.2.10)
\end{document}